% !Mode:: "TeX:UTF-8"

\chapter[哈尔滨工业大学研究生学位论文撰写规范]{哈尔滨工业大学研究生学
  位论文\protect\\撰写规范}[Harbin Institute of Technology Postgraduate Dissertation Writing Specifications]

研究生学位论文是研究生科学研究工作的全面总结,是描述其研究成果、代表其研究水平的
重要学术文献资料,是申请和授予相应学位的基本依据。学位论文撰写是研究生培养过程的
基本训练之一,必须按照确定的规范认真执行。研究生应严肃认真地撰写学位论文,指导教
师应加强指导,严格把关。

学位论文撰写应实事求是,杜绝造假和抄袭等行为;应符合国家及各专业部门制定的有关标
准,符合汉语语法规范。硕士和博士学位论文,除在字数、理论研究的深度及创造性成果等
方面的要求不同外,撰写规范要求基本一致。人文与社会科学、管理学科可在本撰写规范的
基础上补充制定专业的学术规范。

\section{内容要求}[Content specification]
\subsection{题目}[Title]

题目应以简明的词语,恰当、准确、科学地反映论文最重要的特定内容(一般不超过25字),
应中英文对照。题目通常由名词性短语构成,不能含有标点符号;应尽量避免使用不常用的
缩略词、首字母缩写字、字符、代号和公式等。

如题目内容层次很多,难以简化时,可采用题目和副题目相结合的方法。题目与副题目字数
之和不应超过35字,中文的题目与副题目之间用破折号相连,英文则用冒号相连。副题目起
补充、阐明题目的作用。题目和副题目在整篇学位论文中的不同地方出现时,应保持一致。

\subsection{摘要与关键词}[Abstraction and key words]
\subsubsection{摘要}[Abstraction]

摘要是论文内容的高度概括,应具有独立性和自含性,即不阅读论文的全文,就能通过摘要
了解整个论文的必要信息。摘要应包括本论文研究的目的、理论与实际意义、主要研究内容、
研究方法等,重点突出研究成果和结论。

摘要的内容要完整、客观、准确,应做到不遗漏、不拔高、不添加。摘要应按层次逐段简要
写出,避免将摘要写成目录式的内容介绍。摘要在叙述研究内容、研究方法和主要结论时,
除作者的价值和经验判断可以使用第一人称外,一般使用第三人称,采用“分析了……原因”、
“认为……”、“对……进行了探讨”等记述方法进行描述。避免主观性的评价意见,避免
对背景、目的、意义、概念和一般性(常识性)理论叙述过多。

摘要需采用规范的名词术语(包括地名、机构名和人名)。对个别新术语或无中文译文的术
语,可用外文或在中文译文后加括号注明外文。摘要中不宜使用公式、化学结构式、图表、
非常用的缩写词和非公知公用的符号与术语,不标注引用文献编号。

博士学位论文摘要应包括以下几个方面的内容:

(1)论文的研究背景及目的。简洁准确地交代论文的研究背景与意义、相关领域的研究现
状、论文所针对的关键科学问题,使读者把握论文选题的必要性和重要性。此部分介绍不宜
写得过多,一般不多于400字。

(2)论文的主要研究内容。介绍论文所要解决核心问题开展的主要研究工作以及研究方法
或研究手段,使读者可以了解论文的研究思路、研究方案、研究方法或手段的合理性与先进
性。

(3)论文的主要创新成果。简要阐述论文的新思想、新观点、新技术、新方法、新结论等
主要信息,使读者可以了解论文的创新性。

(4)论文成果的理论和实际意义。客观、简要地介绍论文成果的理论和实际意义,使读者
可以快速获得论文的学术价值。

\subsubsection{关键词}[Keywords]
关键词是供检索用的主题词条。关键词应集中体现论文特色,反映研究成果的内涵,具有语
义性,在论文中有明确的出处,并应尽量采用《汉语主题词表》或各专业主题词表提供的规
范词,应列取3$\sim$6个关键词,按词条的外延层次从大到小排列。

\subsection{目录}[Content]

论文中各章节的顺序排列表,包括论文中全部章、节、条三级标题及其页码。

\subsection{论文正文}[Main body]

论文正文包括绪论、论文主体及结论等部分。

\subsubsection{绪论}
绪论一般作为第1章。绪论应包括:本研究课题的来源、背景及其理论意义与实际意义;国
内外与课题相关研究领域的研究进展及成果、存在的不足或有待深入研究的问题,归纳出将
要开展研究的理论分析框架、研究内容、研究程序和方法。

绪论部分要注意对论文所引用国内外文献的准确标注。绪论的主要研究内容的撰写宜使用将
来时态,切忌将论文目录直接作为研究内容。

\subsubsection{论文主体}
论文主体是学位论文的主要部分,应该结构严谨,层次清晰,重点突出,文字简练、通顺。
论文各章之间应该前后关联,构成一个有机的整体。论文给出的数据必须真实可靠,推理正
确,结论明确,无概念性和科学性错误。对于科学实验、计算机仿真的条件、实验过程、仿
真过程等需加以叙述,避免直接给出结果、曲线和结论。引用他人研究成果或采用他人成说
时,应注明出处,不得将其与本人提出的理论分析混淆在一起。

论文主体各章后应有一节“本章小结”,实验方法或材料等章节可不写“本章小结”。各章
小结是对各章研究内容、方法与成果的简洁准确的总结与概括,也是论文最后结论的依据。

\subsubsection{结论}
结论作为学位论文正文的组成部分,单独排写,不加章标题序号,不标注




% Local Variables:
% TeX-master: "../main"
% TeX-engine: xetex
% End: